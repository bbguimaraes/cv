\phantomsection
\section*{Professional experience}
\label{sec:professional}

\phantomsection
\href{https://redhat.com}{Red Hat} (2016-pres.)
\label{subsec:redhat}

\begin{itemize}
    \item
        Worked on the OpenShift
        \href{https://prow.ci.openshift.org}{continuous integration system}, a
        highly parallel, distributed scheduler\slash executor built with
        OpenShift\slash Kubernetes clusters.
    \item
        Supported 800 repositories, 60 organizations, 100s of developers, 200
        compute nodes, 100k jobs\slash week.
    \item
        Designed and implemented the current
        \href
            {https://github.com/openshift/ci-tools/blob/master/pkg/steps/multi_stage.go}
            {test DSL and executor},
        separating complex test scripts into discrete steps, greatly simplifying
        them and making execution more robust and intuitive.
    \item
        Transitioned a mostly legacy code base into a reliable and thoroughly
        tested project, maintained performance of tools and services processing
        large amounts of data through constant analysis and optimization.
    \item
        Wrote extensive,
        \href
            {https://docs.ci.openshift.org/docs/architecture/ci-operator-steps/}
            {internal}\slash \href
            {https://docs.ci.openshift.org/docs/architecture/step-registry}
            {external}
        \href
            {https://docs.ci.openshift.org}
            {documentation}.
    \item
        Entirely free\slash open-source
        \href{https://github.com/bbguimaraes}{work}.
\end{itemize}

\phantomsection
\href{https://ucs.br}{Universidade de Caxias do Sul} (2012-2015)
\label{subsec:ucs}

\begin{itemize}
    \item
        Worked on the infrastructure and applications that served 30k users
        (teachers, students, employees).
    \item
        Web applications (Linux, Python, containers) and services (web servers,
        cache, task queues, databases), tools (CI, ticket system), integrated
        authentication.
\end{itemize}
