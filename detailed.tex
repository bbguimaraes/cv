\noindent
\textsc{\Large Computer technician} \\\\
Technical course \\\\
\textbf{Institution}: Centro Superior de Tecnologia TECBrasil \\
\textbf{Duration}: 1.5 years \\\\

\noindent
\textsc{\Large Linux administration} \\\\
Short course \\\\
\textbf{Original title}: Curso Administração de SO Linux \\
\textbf{Institution}: Universidade de Caxias do Sul \\
\textbf{Instructor}: Amador Pahim \\
\textbf{Duration}: 32h \\
\textbf{Website}: \url{https://github.com/jpteixeira/AdmLinuxEd02} \\\\
\textbf{Topics}:
    \begin{itemize}
        \vspace{-2.5mm}
        \itemsep-1mm
        \item introduction: unix/linux history, free software, linux
            distributions
        \item basic commands: shell commands, file/directory manipulation
        \item vi: text editing
        \item shell script: if/for/case/while, variables, input/output
        \item software installation: rpm/yum, deb/apt, source
        \item system initialization: boot sequence, grub, runlevels, init,
            sysv/upstart
    \end{itemize}
\vspace{5mm}

\noindent
\textsc{\Large Immune System based Data Mining: a case study on Fraud
    Detection} \\\\
Bachelor's degree thesis \\\\
\textbf{Original title}:
    Mineração de Dados baseada nos Sistemas Imunológicos: um estudo de caso
    na Detecção de Fraude \\
\textbf{Institution}: Universidade de Caxias do Sul \\
\textbf{Advisor}: Carine Geltrudes Webber \\
\textbf{Website}: \url{https://github.com/bbguimaraes/tcc} \\\\
\textbf{Abstract}:
    Artificial Immune Systems are a field of Artificial Intelligence that
    developed on the early 90's, and remains subject to researches event today.
    Beginning on computer security systems, algorithms based on the immune
    system have been used on many areas of computing. The power of abstraction
    the design of these systems as components of the natural immune system
    helped their development. Using a package of immune algorithms and the WEKA
    environment, this study will verify how the application of Artificial
    Immune Systems influences a computational system, comparing them with more
    traditional techniques from Data Mining and Artificial Intelligence. \\\\

\noindent
\textsc
    {\Large Efficient liver surgery planning in 3D based on functional segment
    classification and volumetric information} \\\\
Research project \\\\
\textbf{Original title}:
    Hepatectomia 3-D: Visualização e interação 3D aplicados à simulação
    de cirurgia hepática \\
\textbf{Institution}: Universidade de Caxias do Sul / FAPERGS \\
\textbf{Advisor}: Anderson Maciel \\
\textbf{Duration}: 6 months \\
\textbf{Website}:
    \url{http://www.inf.ufrgs.br/~hgdebarba/files/EMBC10_smartliver.pdf} \\
\textbf{Attributions}:
    development of a mass-spring system to simulate the thread of a surgical
    suture on a computer model of a human liver.  \\
\textbf{Technologies}: c/c++, opengl, graphics programming. \\\\
\textbf{Abstract}:
    Anatomic hepatectomies are resections in which compromised segments or
    sectors of the liver are extracted according to the topological structure
    of its vascular elements. Such structure varies considerably among
    patients, which makes the current anatomy-based planning methods often
    inaccurate. In this work we propose a strategy to efficiently and
    semi-automatically segment and classify patient-specific liver models in
    3D. The method is based on standard CT datasets and allows accurate
    estimation of functional remaining liver volume. Experiments showing
    effectiveness of the method are presented, and quantitative and qualitative
    results are discussed. \\\\

\noindent
\textsc{\Large Development of an anti-phishing filter} \\\\
Research project \\\\
\textbf{Institution}: Universidade de Caxias do Sul / CNPq \\
\textbf{Advisor}: Carine Geltrudes Webber \\
\textbf{Duration}: 9 months \\
\textbf{Attributions}:
    integration of an anti-phishing filter based on artificial immune systems
    with the postfix mail server. \\
\textbf{Technologies}:
    postfix, linux, shell script, systems programming, c\#, artificial
    intelligence \\\\

\noindent
\textsc{\Large Universidade de Caxias do Sul} \\\\
Programmer \\\\
\textbf{Duration}: 2 years (ongoing) \\
\textbf{Attributions}:
    development of the computer systems for the university's teachers, students
    and employees (\textasciitilde{}30k users). Web development with a
    traditional linux stack and web server configuration and maintenance.
    Configuration and maintenance of auxiliary web infrastructure, such as the
    memcache memory caching service and the celery python work queue manager.
    Configuration of a redmine instance for internal issue tracking. Itegrated
    authetication (single sign on) of multiple services (internal python/django
    applications, redmine, wordpress). \\
\textbf{Technologies}:
    python, django, html, css, js, svn, git, TDD, apache, nginx, sqlite,
    oracle, memcache, mongodb \\\\

\noindent
\textsc{\Large bash and gnu/linux} \\\\
Talk \\\\
\textbf{Event}: TchêLinux Caxias  2013 \\
\textbf{Summary}:
    A pratical demonstration of how the gnu tools in a linux operating system
    can be used in common day-to-day tasks of programmers/system
    administrators. \\
\textbf{Website}:
    \url{http://tchelinux.org/wiki/evento_2013_08_cxs#bash_e_gnulinux} \\
\textbf{Slides}: \url{https://bbguimaraes.com/talks/} \\\\

\noindent
\textsc{\Large Free cloud with owncloud} \\\\
Talk \\\\
\textbf{Event}: TchêLinux Caxias 2014 \\
\textbf{Summary}:
    A presentation of the project and a practical guide to enable people
    with little or no experience with servers and/or owncloud to install and
    configure this personal cloud service. \\
\textbf{Website}: \url{http://caxias.tchelinux.org#speech-4} \\
\textbf{Slides}: \url{https://bbguimaraes.com/talks/} \\\\

\noindent
\textsc{\Large Linux containers} \\\\
Talk \\\\
\textbf{Event}: TcheLinux Porto Alegre 2014 \\
\textbf{Summary}:
    A presentation of the tools in the kernel and userspace for application
    isolation. Technologies in constant development will be discussed: systemd,
    lxd and docker, with practical demonstrations of their use in systems
    development and discussion of their advantages and disadvantages. \\
\textbf{Website}: \url{http://poa.tchelinux.org/#speech-12} \\
\textbf{Slides}: \url{https://bbguimaraes.com/talks/} \\\\

\noindent
\textsc{\Large Side projects} \\
\begin{itemize}
    \vspace{-2.5mm}
    \itemsep-1mm
    \item Personal owncloud installation on a VPS and migrating this server to
        a docker infrastructure, with separate containers for the filesystem,
        postgresql, php-fpm and nginx.
    \item Experiments integrating linux containers (lxc/docker) on development
        environments at work and on personal projects.
    \item Experiments with a raspberry pi: running services such as ssh,
        stunnel, apache, nginx, php, owncloud, nagios.
    \item Installing and running CyanogenMod on a cellphone.
    \item Running arch linux on all computers (a never-ending adventure in
        itself).
    \item Reading and doing exercises of books from different areas, such as
        \textit{Structure and Interpretation of Computer Programs}, \textit{The
        Linux Programming Interface} and \textit{21st Century C}.
    \item Watching regularly computer related podcasts, such as \textit{The
        Linux Action Show}, \textit{Linux Unplugged} and \textit{TechSNAP}.
    \item Playing keyboard on a blues band and bass guitar on a blues/rock
        trio.
\end{itemize}
