\section*{Research projects}

\subsection*{
    Efficient liver surgery planning in 3D based on functional segment
    classification and volumetric information}

\begin{description}
    \item[Original title]:
        Hepatectomia 3-D: Visualização e interação 3D aplicados à simulação de
        cirurgia hepática
    \item[Institution]: Universidade de Caxias do Sul / FAPERGS
    \item[Advisor]: Anderson Maciel
    \item[Duration]: 6 months
    \item[Attributions]:
        development of a mass-spring system to simulate the thread of a
        surgical suture on a computer model of a human liver.
    \item[Technologies]:
        \begin{itemize}
            \item \textbf{Operating systems}: linux, windows
            \item \textbf{Languages/libraries}: c/c++, opengl
            \item \textbf{Area}: graphics programming, physics
        \end{itemize}
    \item[Abstract]:
        Anatomic hepatectomies are resections in which compromised segments or
        sectors of the liver are extracted according to the topological
        structure of its vascular elements. Such structure varies considerably
        among patients, which makes the current anatomy-based planning methods
        often inaccurate. In this work we propose a strategy to efficiently and
        semi-automatically segment and classify patient-specific liver models
        in 3D. The method is based on standard CT datasets and allows accurate
        estimation of functional remaining liver volume. Experiments showing
        effectiveness of the method are presented, and quantitative and
        qualitative results are discussed.
\end{description}

\subsection*{Development of an anti-phishing filter}

\begin{description}
    \item[Institution]: Universidade de Caxias do Sul / CNPq
    \item[Advisor]: Carine Geltrudes Webber
    \item[Duration]: 9 months
    \item[Attributions]:
        integration of an anti-phishing filter based on artificial immune
        systems with the postfix mail server.
    \item[Technologies]:
        \begin{itemize}
            \item \textbf{Operating system}: linux
            \item \textbf{Languages}: shell, c\#
            \item \textbf{Programs}: postfix
            \item \textbf{Areas}: systems programming, artificial intelligence
        \end{itemize}
\end{description}
