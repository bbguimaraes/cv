\noindent
\textsc{\Huge Research projects} \\\\\\\\
\noindent
\textsc
    {\Large Efficient liver surgery planning in 3D based on functional segment
    classification and volumetric information} \\\\
Research project \\\\
\textbf{Original title}:
    Hepatectomia 3-D: Visualização e interação 3D aplicados à simulação
    de cirurgia hepática \\
\textbf{Institution}: Universidade de Caxias do Sul / FAPERGS \\
\textbf{Advisor}: Anderson Maciel \\
\textbf{Duration}: 6 months \\
\textbf{Website}:
    \url{http://www.inf.ufrgs.br/~hgdebarba/files/EMBC10_smartliver.pdf} \\
\textbf{Attributions}:
    development of a mass-spring system to simulate the thread of a surgical
    suture on a computer model of a human liver.  \\
\textbf{Technologies}: c/c++, opengl, graphics programming. \\\\
\textbf{Abstract}:
    Anatomic hepatectomies are resections in which compromised segments or
    sectors of the liver are extracted according to the topological structure
    of its vascular elements. Such structure varies considerably among
    patients, which makes the current anatomy-based planning methods often
    inaccurate. In this work we propose a strategy to efficiently and
    semi-automatically segment and classify patient-specific liver models in
    3D. The method is based on standard CT datasets and allows accurate
    estimation of functional remaining liver volume. Experiments showing
    effectiveness of the method are presented, and quantitative and qualitative
    results are discussed. \\\\

\noindent
\textsc{\Large Development of an anti-phishing filter} \\\\
Research project \\\\
\textbf{Institution}: Universidade de Caxias do Sul / CNPq \\
\textbf{Advisor}: Carine Geltrudes Webber \\
\textbf{Duration}: 9 months \\
\textbf{Attributions}:
    integration of an anti-phishing filter based on artificial immune systems
    with the postfix mail server. \\
\textbf{Technologies}:
    postfix, linux, shell script, systems programming, c\#, artificial
    intelligence
