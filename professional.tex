\phantomsection
\section*{Experience}
\label{sec:professional}

\subsection*{Universidade de Caxias do Sul}

\begin{description}
    \item[Titles]:
        Trainee programmer (2012-2014), Programmer (2014-2015), Senior
        programmer (2015)
    \item[Location]: Caxias do Sul, Brazil
    \item[Attributions]:
        \begin{itemize}
            \item
                Development of computer systems used by the university's
                teachers, students, and employees ($\sim$30k users).
            \item
                Web development with a traditional linux stack, web server
                configuration and maintenance.
            \item
                Configuration and maintenance of auxiliary web application
                infrastructure, e.g.: memcache (memory caching service), celery
                (python work queue manager), and buildbot (continuous
                integration service).
            \item
                Configuration of a redmine instance for internal issue
                tracking.
            \item
                Integrated authetication (single sign on) of multiple services
                (internal python/django applications, redmine, wordpress).
            \item
                Utilization of linux containers for the development and
                deployment of web applications.
        \end{itemize}
    \item[Technologies]:
        \begin{itemize}
            \item \textbf{Operating systems}: linux (CentOS, RHEL, docker)
            \item \textbf{Languages}: python, html, css, js, c, ruby, php
            \item \textbf{Web frameworks}: django, rails, wordpress
            \item \textbf{Databases}: sqlite, postgresql, oracle
            \item \textbf{Web servers}:
                apache, nginx, mod\_wsgi, uwsgi, passenger
            \item \textbf{Services}:
                memcache, sentry, celery, buildbot, mongodb, ldap, active
                directory, SAML
        \end{itemize}
\end{description}

\phantomsection
\subsection*{Red Hat - Openshift - Integration Services}
\label{subsec:redhat}

\begin{description}
    \item[Title]: Associate software engineer (2016)
    \item[Location]: Brno, Czech Republic (remote team in the USA)
    \item[Attributions]:
        \begin{itemize}
            \item
                Integration of external applications that provide additional
                services to the cluster into the core distribution.
            \item
                Development and maintenance of deployment configuration and
                scripts, automated tests.
            \item
                Transition the deployment of the components from shell scripts
                to ansible playbooks.
        \end{itemize}
    \item[Platform]: kubernetes, openshift, shell, go, java.
    \item[Services]:
        logging (fluentd, elasticsearch, kibana), metrics (heapster, hawkular,
        cassandra), API (apiman).
    \item[Public repositories]:
        \begin{itemize}
            \item \url{https://github.com/openshift/origin-metrics.git}
            \item \url{https://github.com/openshift/origin-aggregated-logging.git}
            \item \url{https://github.com/openshift/openshift-ansible.git}
        \end{itemize}
\end{description}

\subsection*{Red Hat - Openshift - Continuous Integration/Delivery}

\begin{description}
    \item[Titles]:
        Associate software engineer (2017), Software engineer (2017-)
    \item[Location]: Brno, Czech Republic (team local and remote in the USA)
    \item[Attributions]:
        \begin{itemize}
            \item
                Maintenance of the processes for building, verifying and
                publishing official releases and container images.
            \item
                Development of the tools used in continuous integration for
                pull request validation, unit and integration tests, and
                internal tools used by the team.
            \item
                Deployment automation, maintenance, and testing of the
                Openshift Online and Dedicated clusters.
            \item
                Development and deployment of upstream kubernetes CI
                infrastructure.
        \end{itemize}
    \item[Platform]: kubernetes, openshift, jenkins, ansible.
    \item[Infrastructure]: docker, vagrant, bare-metal, Amazon EC2.
    \item[CI jobs]: python, shell, go, groovy, java.
    \item[Public repositories]:
        \begin{itemize}
            \item \url{https://github.com/openshift/aos-cd-jobs.git}
            \item \url{https://github.com/openshift/release.git}
            \item \url{https://github.com/kubernetes/test-infra.git}
            \item \url{https://github.com/openshift/origin-ci-tool.git}
            \item \url{https://github.com/openshift/openshift-ansible.git}
        \end{itemize}
\end{description}
